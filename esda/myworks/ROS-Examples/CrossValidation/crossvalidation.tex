% Options for packages loaded elsewhere
\PassOptionsToPackage{unicode}{hyperref}
\PassOptionsToPackage{hyphens}{url}
%
\documentclass[
]{article}
\usepackage{lmodern}
\usepackage{amssymb,amsmath}
\usepackage{ifxetex,ifluatex}
\ifnum 0\ifxetex 1\fi\ifluatex 1\fi=0 % if pdftex
  \usepackage[T1]{fontenc}
  \usepackage[utf8]{inputenc}
  \usepackage{textcomp} % provide euro and other symbols
\else % if luatex or xetex
  \usepackage{unicode-math}
  \defaultfontfeatures{Scale=MatchLowercase}
  \defaultfontfeatures[\rmfamily]{Ligatures=TeX,Scale=1}
\fi
% Use upquote if available, for straight quotes in verbatim environments
\IfFileExists{upquote.sty}{\usepackage{upquote}}{}
\IfFileExists{microtype.sty}{% use microtype if available
  \usepackage[]{microtype}
  \UseMicrotypeSet[protrusion]{basicmath} % disable protrusion for tt fonts
}{}
\makeatletter
\@ifundefined{KOMAClassName}{% if non-KOMA class
  \IfFileExists{parskip.sty}{%
    \usepackage{parskip}
  }{% else
    \setlength{\parindent}{0pt}
    \setlength{\parskip}{6pt plus 2pt minus 1pt}}
}{% if KOMA class
  \KOMAoptions{parskip=half}}
\makeatother
\usepackage{xcolor}
\IfFileExists{xurl.sty}{\usepackage{xurl}}{} % add URL line breaks if available
\IfFileExists{bookmark.sty}{\usepackage{bookmark}}{\usepackage{hyperref}}
\hypersetup{
  pdftitle={Regression and Other Stories: Cross-validation},
  pdfauthor={Andrew Gelman, Jennifer Hill, Aki Vehtari},
  hidelinks,
  pdfcreator={LaTeX via pandoc}}
\urlstyle{same} % disable monospaced font for URLs
\usepackage[margin=1in]{geometry}
\usepackage{color}
\usepackage{fancyvrb}
\newcommand{\VerbBar}{|}
\newcommand{\VERB}{\Verb[commandchars=\\\{\}]}
\DefineVerbatimEnvironment{Highlighting}{Verbatim}{commandchars=\\\{\}}
% Add ',fontsize=\small' for more characters per line
\usepackage{framed}
\definecolor{shadecolor}{RGB}{248,248,248}
\newenvironment{Shaded}{\begin{snugshade}}{\end{snugshade}}
\newcommand{\AlertTok}[1]{\textcolor[rgb]{0.94,0.16,0.16}{#1}}
\newcommand{\AnnotationTok}[1]{\textcolor[rgb]{0.56,0.35,0.01}{\textbf{\textit{#1}}}}
\newcommand{\AttributeTok}[1]{\textcolor[rgb]{0.77,0.63,0.00}{#1}}
\newcommand{\BaseNTok}[1]{\textcolor[rgb]{0.00,0.00,0.81}{#1}}
\newcommand{\BuiltInTok}[1]{#1}
\newcommand{\CharTok}[1]{\textcolor[rgb]{0.31,0.60,0.02}{#1}}
\newcommand{\CommentTok}[1]{\textcolor[rgb]{0.56,0.35,0.01}{\textit{#1}}}
\newcommand{\CommentVarTok}[1]{\textcolor[rgb]{0.56,0.35,0.01}{\textbf{\textit{#1}}}}
\newcommand{\ConstantTok}[1]{\textcolor[rgb]{0.00,0.00,0.00}{#1}}
\newcommand{\ControlFlowTok}[1]{\textcolor[rgb]{0.13,0.29,0.53}{\textbf{#1}}}
\newcommand{\DataTypeTok}[1]{\textcolor[rgb]{0.13,0.29,0.53}{#1}}
\newcommand{\DecValTok}[1]{\textcolor[rgb]{0.00,0.00,0.81}{#1}}
\newcommand{\DocumentationTok}[1]{\textcolor[rgb]{0.56,0.35,0.01}{\textbf{\textit{#1}}}}
\newcommand{\ErrorTok}[1]{\textcolor[rgb]{0.64,0.00,0.00}{\textbf{#1}}}
\newcommand{\ExtensionTok}[1]{#1}
\newcommand{\FloatTok}[1]{\textcolor[rgb]{0.00,0.00,0.81}{#1}}
\newcommand{\FunctionTok}[1]{\textcolor[rgb]{0.00,0.00,0.00}{#1}}
\newcommand{\ImportTok}[1]{#1}
\newcommand{\InformationTok}[1]{\textcolor[rgb]{0.56,0.35,0.01}{\textbf{\textit{#1}}}}
\newcommand{\KeywordTok}[1]{\textcolor[rgb]{0.13,0.29,0.53}{\textbf{#1}}}
\newcommand{\NormalTok}[1]{#1}
\newcommand{\OperatorTok}[1]{\textcolor[rgb]{0.81,0.36,0.00}{\textbf{#1}}}
\newcommand{\OtherTok}[1]{\textcolor[rgb]{0.56,0.35,0.01}{#1}}
\newcommand{\PreprocessorTok}[1]{\textcolor[rgb]{0.56,0.35,0.01}{\textit{#1}}}
\newcommand{\RegionMarkerTok}[1]{#1}
\newcommand{\SpecialCharTok}[1]{\textcolor[rgb]{0.00,0.00,0.00}{#1}}
\newcommand{\SpecialStringTok}[1]{\textcolor[rgb]{0.31,0.60,0.02}{#1}}
\newcommand{\StringTok}[1]{\textcolor[rgb]{0.31,0.60,0.02}{#1}}
\newcommand{\VariableTok}[1]{\textcolor[rgb]{0.00,0.00,0.00}{#1}}
\newcommand{\VerbatimStringTok}[1]{\textcolor[rgb]{0.31,0.60,0.02}{#1}}
\newcommand{\WarningTok}[1]{\textcolor[rgb]{0.56,0.35,0.01}{\textbf{\textit{#1}}}}
\usepackage{graphicx,grffile}
\makeatletter
\def\maxwidth{\ifdim\Gin@nat@width>\linewidth\linewidth\else\Gin@nat@width\fi}
\def\maxheight{\ifdim\Gin@nat@height>\textheight\textheight\else\Gin@nat@height\fi}
\makeatother
% Scale images if necessary, so that they will not overflow the page
% margins by default, and it is still possible to overwrite the defaults
% using explicit options in \includegraphics[width, height, ...]{}
\setkeys{Gin}{width=\maxwidth,height=\maxheight,keepaspectratio}
% Set default figure placement to htbp
\makeatletter
\def\fps@figure{htbp}
\makeatother
\setlength{\emergencystretch}{3em} % prevent overfull lines
\providecommand{\tightlist}{%
  \setlength{\itemsep}{0pt}\setlength{\parskip}{0pt}}
\setcounter{secnumdepth}{-\maxdimen} % remove section numbering

\title{Regression and Other Stories: Cross-validation}
\author{Andrew Gelman, Jennifer Hill, Aki Vehtari}
\date{2020-09-10}

\begin{document}
\maketitle

{
\setcounter{tocdepth}{2}
\tableofcontents
}
Introduction to cross-validation for linear regression. See Chapter 11
in Regression and Other Stories.

\begin{center}\rule{0.5\linewidth}{0.5pt}\end{center}

\hypertarget{load-packages}{%
\paragraph{Load packages}\label{load-packages}}

\begin{Shaded}
\begin{Highlighting}[]
\KeywordTok{library}\NormalTok{(}\StringTok{"rprojroot"}\NormalTok{)}
\NormalTok{root<-}\KeywordTok{has_file}\NormalTok{(}\StringTok{".ROS-Examples-root"}\NormalTok{)}\OperatorTok{$}\KeywordTok{make_fix_file}\NormalTok{()}
\KeywordTok{library}\NormalTok{(}\StringTok{"rstanarm"}\NormalTok{)}
\KeywordTok{library}\NormalTok{(}\StringTok{"loo"}\NormalTok{)}
\KeywordTok{library}\NormalTok{(}\StringTok{"ggplot2"}\NormalTok{)}
\KeywordTok{library}\NormalTok{(}\StringTok{"bayesplot"}\NormalTok{)}
\KeywordTok{theme_set}\NormalTok{(bayesplot}\OperatorTok{::}\KeywordTok{theme_default}\NormalTok{(}\DataTypeTok{base_family =} \StringTok{"sans"}\NormalTok{, }\DataTypeTok{base_size=}\DecValTok{16}\NormalTok{))}
\KeywordTok{library}\NormalTok{(}\StringTok{"foreign"}\NormalTok{)}
\CommentTok{# for reproducibility}
\NormalTok{SEED <-}\StringTok{ }\DecValTok{1507}
\end{Highlighting}
\end{Shaded}

\hypertarget{simulation-data-example}{%
\subsection{Simulation data example}\label{simulation-data-example}}

\hypertarget{simulate-fake-data}{%
\paragraph{Simulate fake data}\label{simulate-fake-data}}

\begin{Shaded}
\begin{Highlighting}[]
\NormalTok{x <-}\StringTok{ }\DecValTok{1}\OperatorTok{:}\DecValTok{20}
\NormalTok{n <-}\StringTok{ }\KeywordTok{length}\NormalTok{(x)}
\NormalTok{a <-}\StringTok{ }\FloatTok{0.2}
\NormalTok{b <-}\StringTok{ }\FloatTok{0.3}
\NormalTok{sigma <-}\StringTok{ }\DecValTok{1}
\CommentTok{# set the random seed to get reproducible results}
\CommentTok{# change the seed to experiment with variation due to random noise}
\KeywordTok{set.seed}\NormalTok{(}\DecValTok{2141}\NormalTok{) }
\NormalTok{y <-}\StringTok{ }\NormalTok{a }\OperatorTok{+}\StringTok{ }\NormalTok{b}\OperatorTok{*}\NormalTok{x }\OperatorTok{+}\StringTok{ }\NormalTok{sigma}\OperatorTok{*}\KeywordTok{rnorm}\NormalTok{(n)}
\NormalTok{fake <-}\StringTok{ }\KeywordTok{data.frame}\NormalTok{(x, y)}
\end{Highlighting}
\end{Shaded}

\hypertarget{fit-linear-model}{%
\paragraph{Fit linear model}\label{fit-linear-model}}

\begin{Shaded}
\begin{Highlighting}[]
\NormalTok{fit_all <-}\StringTok{ }\KeywordTok{stan_glm}\NormalTok{(y }\OperatorTok{~}\StringTok{ }\NormalTok{x, }\DataTypeTok{data =}\NormalTok{ fake, }\DataTypeTok{seed=}\DecValTok{2141}\NormalTok{, }\DataTypeTok{chains=}\DecValTok{10}\NormalTok{, }\DataTypeTok{refresh=}\DecValTok{0}\NormalTok{)}
\end{Highlighting}
\end{Shaded}

\hypertarget{fit-linear-model-without-18th-observation}{%
\paragraph{Fit linear model without 18th
observation}\label{fit-linear-model-without-18th-observation}}

\begin{Shaded}
\begin{Highlighting}[]
\NormalTok{fit_minus_}\DecValTok{18}\NormalTok{ <-}\StringTok{ }\KeywordTok{stan_glm}\NormalTok{(y }\OperatorTok{~}\StringTok{ }\NormalTok{x, }\DataTypeTok{data =}\NormalTok{ fake[}\OperatorTok{-}\DecValTok{18}\NormalTok{,], }\DataTypeTok{seed=}\DecValTok{2141}\NormalTok{, }\DataTypeTok{refresh=}\DecValTok{0}\NormalTok{)}
\end{Highlighting}
\end{Shaded}

\hypertarget{extract-posterior-draws}{%
\paragraph{Extract posterior draws}\label{extract-posterior-draws}}

\begin{Shaded}
\begin{Highlighting}[]
\NormalTok{sims <-}\StringTok{ }\KeywordTok{as.matrix}\NormalTok{(fit_all)}
\NormalTok{sims_minus_}\DecValTok{18}\NormalTok{ <-}\StringTok{ }\KeywordTok{as.matrix}\NormalTok{(fit_minus_}\DecValTok{18}\NormalTok{)}
\end{Highlighting}
\end{Shaded}

\hypertarget{compute-posterior-predictive-distribution-given-x18}{%
\paragraph{Compute posterior predictive distribution given
x=18}\label{compute-posterior-predictive-distribution-given-x18}}

\begin{Shaded}
\begin{Highlighting}[]
\NormalTok{condpred<-}\KeywordTok{data.frame}\NormalTok{(}\DataTypeTok{y=}\KeywordTok{seq}\NormalTok{(}\DecValTok{0}\NormalTok{,}\DecValTok{9}\NormalTok{,}\DataTypeTok{length.out=}\DecValTok{100}\NormalTok{))}
\NormalTok{condpred}\OperatorTok{$}\NormalTok{x <-}\StringTok{ }\KeywordTok{sapply}\NormalTok{(condpred}\OperatorTok{$}\NormalTok{y, }\DataTypeTok{FUN=}\ControlFlowTok{function}\NormalTok{(y) }\KeywordTok{mean}\NormalTok{(}\KeywordTok{dnorm}\NormalTok{(y, sims[,}\DecValTok{1}\NormalTok{] }\OperatorTok{+}\StringTok{ }\NormalTok{sims[,}\DecValTok{2}\NormalTok{]}\OperatorTok{*}\DecValTok{18}\NormalTok{, sims[,}\DecValTok{3}\NormalTok{])}\OperatorTok{*}\DecValTok{6}\OperatorTok{+}\DecValTok{18}\NormalTok{))}
\end{Highlighting}
\end{Shaded}

\hypertarget{compute-loo-posterior-predictive-distribution-given-x18}{%
\paragraph{Compute LOO posterior predictive distribution given
x=18}\label{compute-loo-posterior-predictive-distribution-given-x18}}

\begin{Shaded}
\begin{Highlighting}[]
\NormalTok{condpredloo<-}\KeywordTok{data.frame}\NormalTok{(}\DataTypeTok{y=}\KeywordTok{seq}\NormalTok{(}\DecValTok{0}\NormalTok{,}\DecValTok{9}\NormalTok{,}\DataTypeTok{length.out=}\DecValTok{100}\NormalTok{))}
\NormalTok{condpredloo}\OperatorTok{$}\NormalTok{x <-}\StringTok{ }\KeywordTok{sapply}\NormalTok{(condpredloo}\OperatorTok{$}\NormalTok{y, }\DataTypeTok{FUN=}\ControlFlowTok{function}\NormalTok{(y) }\KeywordTok{mean}\NormalTok{(}\KeywordTok{dnorm}\NormalTok{(y, sims_minus_}\DecValTok{18}\NormalTok{[,}\DecValTok{1}\NormalTok{] }\OperatorTok{+}\StringTok{ }\NormalTok{sims_minus_}\DecValTok{18}\NormalTok{[,}\DecValTok{2}\NormalTok{]}\OperatorTok{*}\DecValTok{18}\NormalTok{, sims_minus_}\DecValTok{18}\NormalTok{[,}\DecValTok{3}\NormalTok{])}\OperatorTok{*}\DecValTok{6}\OperatorTok{+}\DecValTok{18}\NormalTok{))}
\end{Highlighting}
\end{Shaded}

\hypertarget{create-a-plot-with-posterior-mean-and-posterior-predictive-distribution}{%
\paragraph{Create a plot with posterior mean and posterior predictive
distribution}\label{create-a-plot-with-posterior-mean-and-posterior-predictive-distribution}}

\begin{Shaded}
\begin{Highlighting}[]
\NormalTok{p1 <-}\StringTok{ }\KeywordTok{ggplot}\NormalTok{(fake, }\KeywordTok{aes}\NormalTok{(}\DataTypeTok{x =}\NormalTok{ x, }\DataTypeTok{y =}\NormalTok{ y)) }\OperatorTok{+}
\StringTok{  }\KeywordTok{geom_point}\NormalTok{(}\DataTypeTok{color =} \StringTok{"white"}\NormalTok{, }\DataTypeTok{size =} \DecValTok{3}\NormalTok{) }\OperatorTok{+}
\StringTok{  }\KeywordTok{geom_point}\NormalTok{(}\DataTypeTok{color =} \StringTok{"black"}\NormalTok{, }\DataTypeTok{size =} \DecValTok{2}\NormalTok{)}
\NormalTok{p2 <-}\StringTok{ }\NormalTok{p1 }\OperatorTok{+}
\StringTok{  }\KeywordTok{geom_abline}\NormalTok{(}
    \DataTypeTok{intercept =} \KeywordTok{mean}\NormalTok{(sims[, }\DecValTok{1}\NormalTok{]),}
    \DataTypeTok{slope =} \KeywordTok{mean}\NormalTok{(sims[, }\DecValTok{2}\NormalTok{]),}
    \DataTypeTok{size =} \DecValTok{1}\NormalTok{,}
    \DataTypeTok{color =} \StringTok{"black"}
\NormalTok{  )}
\NormalTok{p3 <-}\StringTok{ }\NormalTok{p2 }\OperatorTok{+}\StringTok{ }
\StringTok{  }\KeywordTok{geom_path}\NormalTok{(}\DataTypeTok{data=}\NormalTok{condpred,}\KeywordTok{aes}\NormalTok{(}\DataTypeTok{x=}\NormalTok{x,}\DataTypeTok{y=}\NormalTok{y), }\DataTypeTok{color=}\StringTok{"black"}\NormalTok{) }\OperatorTok{+}
\StringTok{  }\KeywordTok{geom_vline}\NormalTok{(}\DataTypeTok{xintercept=}\DecValTok{18}\NormalTok{, }\DataTypeTok{linetype=}\DecValTok{3}\NormalTok{, }\DataTypeTok{color=}\StringTok{"grey"}\NormalTok{)}
\end{Highlighting}
\end{Shaded}

\hypertarget{add-loo-mean-and-loo-predictive-distribution-when-x18-is-left-out}{%
\paragraph{Add LOO mean and LOO predictive distribution when x=18 is
left
out}\label{add-loo-mean-and-loo-predictive-distribution-when-x18-is-left-out}}

\begin{Shaded}
\begin{Highlighting}[]
\NormalTok{p4 <-}\StringTok{ }\NormalTok{p3 }\OperatorTok{+}
\StringTok{  }\KeywordTok{geom_point}\NormalTok{(}\DataTypeTok{data=}\NormalTok{fake[}\DecValTok{18}\NormalTok{,], }\DataTypeTok{color =} \StringTok{"grey50"}\NormalTok{, }\DataTypeTok{size =} \DecValTok{5}\NormalTok{, }\DataTypeTok{shape=}\DecValTok{1}\NormalTok{) }\OperatorTok{+}
\StringTok{  }\KeywordTok{geom_abline}\NormalTok{(}
    \DataTypeTok{intercept =} \KeywordTok{mean}\NormalTok{(sims_minus_}\DecValTok{18}\NormalTok{[, }\DecValTok{1}\NormalTok{]),}
    \DataTypeTok{slope =} \KeywordTok{mean}\NormalTok{(sims_minus_}\DecValTok{18}\NormalTok{[, }\DecValTok{2}\NormalTok{]),}
    \DataTypeTok{size =} \DecValTok{1}\NormalTok{,}
    \DataTypeTok{color =} \StringTok{"grey50"}\NormalTok{,}
    \DataTypeTok{linetype=}\DecValTok{2}
\NormalTok{  ) }\OperatorTok{+}
\StringTok{  }\KeywordTok{geom_path}\NormalTok{(}\DataTypeTok{data=}\NormalTok{condpredloo,}\KeywordTok{aes}\NormalTok{(}\DataTypeTok{x=}\NormalTok{x,}\DataTypeTok{y=}\NormalTok{y), }\DataTypeTok{color=}\StringTok{"grey50"}\NormalTok{, }\DataTypeTok{linetype=}\DecValTok{2}\NormalTok{)}
\NormalTok{p4}
\end{Highlighting}
\end{Shaded}

\includegraphics{crossvalidation_files/figure-latex/unnamed-chunk-9-1.pdf}

\hypertarget{compute-posterior-and-loo-residuals}{%
\paragraph{Compute posterior and LOO
residuals}\label{compute-posterior-and-loo-residuals}}

\texttt{loo\_predict()} computes mean of LOO predictive distribution.

\begin{Shaded}
\begin{Highlighting}[]
\NormalTok{fake}\OperatorTok{$}\NormalTok{residual <-}\StringTok{ }\NormalTok{fake}\OperatorTok{$}\NormalTok{y}\OperatorTok{-}\NormalTok{fit_all}\OperatorTok{$}\NormalTok{fitted}
\NormalTok{fake}\OperatorTok{$}\NormalTok{looresidual <-}\StringTok{ }\NormalTok{fake}\OperatorTok{$}\NormalTok{y}\OperatorTok{-}\KeywordTok{loo_predict}\NormalTok{(fit_all)}\OperatorTok{$}\NormalTok{value}
\end{Highlighting}
\end{Shaded}

\hypertarget{plot-posterior-and-loo-residuals}{%
\paragraph{Plot posterior and LOO
residuals}\label{plot-posterior-and-loo-residuals}}

\begin{Shaded}
\begin{Highlighting}[]
\NormalTok{p1 <-}\StringTok{ }\KeywordTok{ggplot}\NormalTok{(fake, }\KeywordTok{aes}\NormalTok{(}\DataTypeTok{x =}\NormalTok{ x, }\DataTypeTok{y =}\NormalTok{ residual)) }\OperatorTok{+}
\StringTok{  }\KeywordTok{geom_point}\NormalTok{(}\DataTypeTok{color =} \StringTok{"black"}\NormalTok{, }\DataTypeTok{size =} \DecValTok{2}\NormalTok{, }\DataTypeTok{shape=}\DecValTok{16}\NormalTok{) }\OperatorTok{+}
\StringTok{  }\KeywordTok{geom_point}\NormalTok{(}\KeywordTok{aes}\NormalTok{(}\DataTypeTok{y=}\NormalTok{looresidual), }\DataTypeTok{color =} \StringTok{"grey50"}\NormalTok{, }\DataTypeTok{size =} \DecValTok{2}\NormalTok{, }\DataTypeTok{shape=}\DecValTok{1}\NormalTok{) }\OperatorTok{+}
\StringTok{  }\KeywordTok{geom_segment}\NormalTok{(}\KeywordTok{aes}\NormalTok{(}\DataTypeTok{xend=}\NormalTok{x, }\DataTypeTok{y=}\NormalTok{residual, }\DataTypeTok{yend=}\NormalTok{looresidual)) }\OperatorTok{+}
\StringTok{  }\KeywordTok{geom_hline}\NormalTok{(}\DataTypeTok{yintercept=}\DecValTok{0}\NormalTok{, }\DataTypeTok{linetype=}\DecValTok{2}\NormalTok{)}
\NormalTok{p1}
\end{Highlighting}
\end{Shaded}

\includegraphics{crossvalidation_files/figure-latex/unnamed-chunk-13-1.pdf}

\hypertarget{standard-deviations-of-posterior-and-loo-residuals}{%
\paragraph{Standard deviations of posterior and LOO
residuals}\label{standard-deviations-of-posterior-and-loo-residuals}}

\begin{Shaded}
\begin{Highlighting}[]
\KeywordTok{round}\NormalTok{(}\KeywordTok{sd}\NormalTok{(fake}\OperatorTok{$}\NormalTok{residual),}\DecValTok{2}\NormalTok{)}
\end{Highlighting}
\end{Shaded}

\begin{verbatim}
[1] 0.92
\end{verbatim}

\begin{Shaded}
\begin{Highlighting}[]
\KeywordTok{round}\NormalTok{(}\KeywordTok{sd}\NormalTok{(fake}\OperatorTok{$}\NormalTok{looresidual),}\DecValTok{2}\NormalTok{)}
\end{Highlighting}
\end{Shaded}

\begin{verbatim}
[1] 1.02
\end{verbatim}

Variance of residuals is connected to R\^{}2, which can be defined as
1-var(res)/var(y)

\begin{Shaded}
\begin{Highlighting}[]
\KeywordTok{round}\NormalTok{(}\DecValTok{1}\OperatorTok{-}\KeywordTok{var}\NormalTok{(fake}\OperatorTok{$}\NormalTok{residual)}\OperatorTok{/}\KeywordTok{var}\NormalTok{(y),}\DecValTok{2}\NormalTok{)}
\end{Highlighting}
\end{Shaded}

\begin{verbatim}
[1] 0.74
\end{verbatim}

\begin{Shaded}
\begin{Highlighting}[]
\KeywordTok{round}\NormalTok{(}\DecValTok{1}\OperatorTok{-}\KeywordTok{var}\NormalTok{(fake}\OperatorTok{$}\NormalTok{looresidual)}\OperatorTok{/}\KeywordTok{var}\NormalTok{(y),}\DecValTok{2}\NormalTok{)}
\end{Highlighting}
\end{Shaded}

\begin{verbatim}
[1] 0.68
\end{verbatim}

\hypertarget{compute-log-posterior-predictive-densities}{%
\paragraph{Compute log posterior predictive
densities}\label{compute-log-posterior-predictive-densities}}

\texttt{log\_lik} returns \(\log(p(y_i|\theta^{(s)}))\)

\begin{Shaded}
\begin{Highlighting}[]
\NormalTok{ll_}\DecValTok{1}\NormalTok{ <-}\StringTok{ }\KeywordTok{log_lik}\NormalTok{(fit_all)}
\end{Highlighting}
\end{Shaded}

compute \(\log(\frac{1}{S}\sum_{s=1}^S p(y_i|\theta^{(s)})\) in
computationally stable way

\begin{Shaded}
\begin{Highlighting}[]
\NormalTok{fake}\OperatorTok{$}\NormalTok{lpd_post <-}\StringTok{ }\NormalTok{matrixStats}\OperatorTok{::}\KeywordTok{colLogSumExps}\NormalTok{(ll_}\DecValTok{1}\NormalTok{) }\OperatorTok{-}\StringTok{ }\KeywordTok{log}\NormalTok{(}\KeywordTok{nrow}\NormalTok{(ll_}\DecValTok{1}\NormalTok{))}
\end{Highlighting}
\end{Shaded}

\hypertarget{compute-log-loo-predictive-densities}{%
\paragraph{Compute log LOO predictive
densities}\label{compute-log-loo-predictive-densities}}

\texttt{loo} uses fast approximate leave-one-out cross-validation

\begin{Shaded}
\begin{Highlighting}[]
\NormalTok{loo_}\DecValTok{1}\NormalTok{ <-}\StringTok{ }\KeywordTok{loo}\NormalTok{(fit_all)}
\NormalTok{fake}\OperatorTok{$}\NormalTok{lpd_loo <-loo_}\DecValTok{1}\OperatorTok{$}\NormalTok{pointwise[,}\StringTok{"elpd_loo"}\NormalTok{]}
\end{Highlighting}
\end{Shaded}

\hypertarget{plot-posterior-and-loo-log-predictive-densities}{%
\paragraph{Plot posterior and LOO log predictive
densities}\label{plot-posterior-and-loo-log-predictive-densities}}

\begin{Shaded}
\begin{Highlighting}[]
\NormalTok{p1 <-}\StringTok{ }\KeywordTok{ggplot}\NormalTok{(fake, }\KeywordTok{aes}\NormalTok{(}\DataTypeTok{x =}\NormalTok{ x, }\DataTypeTok{y =}\NormalTok{ lpd_post)) }\OperatorTok{+}
\StringTok{  }\KeywordTok{geom_point}\NormalTok{(}\DataTypeTok{color =} \StringTok{"black"}\NormalTok{, }\DataTypeTok{size =} \DecValTok{2}\NormalTok{, }\DataTypeTok{shape=}\DecValTok{16}\NormalTok{) }\OperatorTok{+}
\StringTok{  }\KeywordTok{geom_point}\NormalTok{(}\KeywordTok{aes}\NormalTok{(}\DataTypeTok{y=}\NormalTok{lpd_loo), }\DataTypeTok{color =} \StringTok{"grey50"}\NormalTok{, }\DataTypeTok{size =} \DecValTok{2}\NormalTok{, }\DataTypeTok{shape=}\DecValTok{1}\NormalTok{) }\OperatorTok{+}
\StringTok{  }\KeywordTok{geom_segment}\NormalTok{(}\KeywordTok{aes}\NormalTok{(}\DataTypeTok{xend=}\NormalTok{x, }\DataTypeTok{y=}\NormalTok{lpd_post, }\DataTypeTok{yend=}\NormalTok{lpd_loo)) }\OperatorTok{+}
\StringTok{  }\KeywordTok{ylab}\NormalTok{(}\StringTok{"log predictive density"}\NormalTok{)}
\NormalTok{p1}
\end{Highlighting}
\end{Shaded}

\includegraphics{crossvalidation_files/figure-latex/unnamed-chunk-21-1.pdf}

\hypertarget{kidiq-example}{%
\subsection{KidIQ example}\label{kidiq-example}}

\hypertarget{load-childrens-test-scores-data}{%
\paragraph{Load children's test scores
data}\label{load-childrens-test-scores-data}}

\begin{Shaded}
\begin{Highlighting}[]
\NormalTok{kidiq <-}\StringTok{ }\KeywordTok{read.dta}\NormalTok{(}\DataTypeTok{file=}\KeywordTok{root}\NormalTok{(}\StringTok{"KidIQ/data"}\NormalTok{,}\StringTok{"kidiq.dta"}\NormalTok{))}
\end{Highlighting}
\end{Shaded}

\hypertarget{linear-regression}{%
\paragraph{Linear regression}\label{linear-regression}}

\begin{Shaded}
\begin{Highlighting}[]
\NormalTok{fit_}\DecValTok{3}\NormalTok{ <-}\StringTok{ }\KeywordTok{stan_glm}\NormalTok{(kid_score }\OperatorTok{~}\StringTok{ }\NormalTok{mom_hs }\OperatorTok{+}\StringTok{ }\NormalTok{mom_iq, }\DataTypeTok{data=}\NormalTok{kidiq,}
                  \DataTypeTok{seed=}\NormalTok{SEED, }\DataTypeTok{refresh =} \DecValTok{0}\NormalTok{)}
\KeywordTok{print}\NormalTok{(fit_}\DecValTok{3}\NormalTok{)}
\end{Highlighting}
\end{Shaded}

\begin{verbatim}
stan_glm
 family:       gaussian [identity]
 formula:      kid_score ~ mom_hs + mom_iq
 observations: 434
 predictors:   3
------
            Median MAD_SD
(Intercept) 25.8    5.9  
mom_hs       5.9    2.2  
mom_iq       0.6    0.1  

Auxiliary parameter(s):
      Median MAD_SD
sigma 18.2    0.6  

------
* For help interpreting the printed output see ?print.stanreg
* For info on the priors used see ?prior_summary.stanreg
\end{verbatim}

\hypertarget{compute-r2-and-loo-r2-manually}{%
\paragraph{Compute R\^{}2 and LOO-R\^{}2
manually}\label{compute-r2-and-loo-r2-manually}}

\begin{Shaded}
\begin{Highlighting}[]
\NormalTok{respost <-}\StringTok{ }\NormalTok{kidiq}\OperatorTok{$}\NormalTok{kid_score}\OperatorTok{-}\NormalTok{fit_}\DecValTok{3}\OperatorTok{$}\NormalTok{fitted}
\NormalTok{resloo <-}\StringTok{ }\NormalTok{kidiq}\OperatorTok{$}\NormalTok{kid_score}\OperatorTok{-}\KeywordTok{loo_predict}\NormalTok{(fit_}\DecValTok{3}\NormalTok{)}\OperatorTok{$}\NormalTok{value}
\KeywordTok{round}\NormalTok{(R2 <-}\StringTok{ }\DecValTok{1} \OperatorTok{-}\StringTok{ }\KeywordTok{var}\NormalTok{(respost)}\OperatorTok{/}\KeywordTok{var}\NormalTok{(kidiq}\OperatorTok{$}\NormalTok{kid_score), }\DecValTok{2}\NormalTok{)}
\end{Highlighting}
\end{Shaded}

\begin{verbatim}
[1] 0.21
\end{verbatim}

\begin{Shaded}
\begin{Highlighting}[]
\KeywordTok{round}\NormalTok{(R2loo <-}\StringTok{ }\DecValTok{1} \OperatorTok{-}\StringTok{ }\KeywordTok{var}\NormalTok{(resloo)}\OperatorTok{/}\KeywordTok{var}\NormalTok{(kidiq}\OperatorTok{$}\NormalTok{kid_score), }\DecValTok{2}\NormalTok{)}
\end{Highlighting}
\end{Shaded}

\begin{verbatim}
[1] 0.2
\end{verbatim}

\hypertarget{add-five-pure-noise-predictors-to-the-data}{%
\paragraph{Add five pure noise predictors to the
data}\label{add-five-pure-noise-predictors-to-the-data}}

\begin{Shaded}
\begin{Highlighting}[]
\KeywordTok{set.seed}\NormalTok{(SEED)}
\NormalTok{n <-}\StringTok{ }\KeywordTok{nrow}\NormalTok{(kidiq)}
\NormalTok{kidiqr <-}\StringTok{ }\NormalTok{kidiq}
\NormalTok{kidiqr}\OperatorTok{$}\NormalTok{noise <-}\StringTok{ }\KeywordTok{array}\NormalTok{(}\KeywordTok{rnorm}\NormalTok{(}\DecValTok{5}\OperatorTok{*}\NormalTok{n), }\KeywordTok{c}\NormalTok{(n,}\DecValTok{5}\NormalTok{))}
\end{Highlighting}
\end{Shaded}

\hypertarget{linear-regression-with-additional-noise-predictors}{%
\paragraph{Linear regression with additional noise
predictors}\label{linear-regression-with-additional-noise-predictors}}

\begin{Shaded}
\begin{Highlighting}[]
\NormalTok{fit_3n <-}\StringTok{ }\KeywordTok{stan_glm}\NormalTok{(kid_score }\OperatorTok{~}\StringTok{ }\NormalTok{mom_hs }\OperatorTok{+}\StringTok{ }\NormalTok{mom_iq }\OperatorTok{+}\StringTok{ }\NormalTok{noise, }\DataTypeTok{data=}\NormalTok{kidiqr,}
                   \DataTypeTok{seed=}\NormalTok{SEED, }\DataTypeTok{refresh =} \DecValTok{0}\NormalTok{)}
\KeywordTok{print}\NormalTok{(fit_3n)}
\end{Highlighting}
\end{Shaded}

\begin{verbatim}
stan_glm
 family:       gaussian [identity]
 formula:      kid_score ~ mom_hs + mom_iq + noise
 observations: 434
 predictors:   8
------
            Median MAD_SD
(Intercept) 25.4    5.8  
mom_hs       6.0    2.3  
mom_iq       0.6    0.1  
noise1      -0.1    0.9  
noise2      -1.3    0.9  
noise3       0.0    0.9  
noise4       0.2    0.9  
noise5      -0.5    0.9  

Auxiliary parameter(s):
      Median MAD_SD
sigma 18.2    0.6  

------
* For help interpreting the printed output see ?print.stanreg
* For info on the priors used see ?prior_summary.stanreg
\end{verbatim}

\hypertarget{compute-r2-and-loo-r2-manually-1}{%
\paragraph{Compute R\^{}2 and LOO-R\^{}2
manually}\label{compute-r2-and-loo-r2-manually-1}}

\begin{Shaded}
\begin{Highlighting}[]
\NormalTok{respostn <-}\StringTok{ }\NormalTok{kidiq}\OperatorTok{$}\NormalTok{kid_score}\OperatorTok{-}\NormalTok{fit_3n}\OperatorTok{$}\NormalTok{fitted}
\NormalTok{resloon <-}\StringTok{ }\NormalTok{kidiq}\OperatorTok{$}\NormalTok{kid_score}\OperatorTok{-}\KeywordTok{loo_predict}\NormalTok{(fit_3n)}\OperatorTok{$}\NormalTok{value}
\KeywordTok{round}\NormalTok{(R2n <-}\StringTok{ }\DecValTok{1} \OperatorTok{-}\StringTok{ }\KeywordTok{var}\NormalTok{(respostn)}\OperatorTok{/}\KeywordTok{var}\NormalTok{(kidiq}\OperatorTok{$}\NormalTok{kid_score), }\DecValTok{2}\NormalTok{)}
\end{Highlighting}
\end{Shaded}

\begin{verbatim}
[1] 0.22
\end{verbatim}

\begin{Shaded}
\begin{Highlighting}[]
\KeywordTok{round}\NormalTok{(R2loon <-}\StringTok{ }\DecValTok{1} \OperatorTok{-}\StringTok{ }\KeywordTok{var}\NormalTok{(resloon)}\OperatorTok{/}\KeywordTok{var}\NormalTok{(kidiq}\OperatorTok{$}\NormalTok{kid_score), }\DecValTok{2}\NormalTok{)}
\end{Highlighting}
\end{Shaded}

\begin{verbatim}
[1] 0.19
\end{verbatim}

\hypertarget{alternative-more-informative-regularized-horseshoe-prior}{%
\paragraph{Alternative more informative regularized horseshoe
prior}\label{alternative-more-informative-regularized-horseshoe-prior}}

\begin{Shaded}
\begin{Highlighting}[]
\NormalTok{fit_3rhs <-}\StringTok{ }\KeywordTok{stan_glm}\NormalTok{(kid_score }\OperatorTok{~}\StringTok{ }\NormalTok{mom_hs }\OperatorTok{+}\StringTok{ }\NormalTok{mom_iq, }\DataTypeTok{prior=}\KeywordTok{hs}\NormalTok{(), }\DataTypeTok{data=}\NormalTok{kidiq,}
                     \DataTypeTok{seed=}\NormalTok{SEED, }\DataTypeTok{refresh =} \DecValTok{0}\NormalTok{)}
\NormalTok{fit_3rhsn <-}\StringTok{ }\KeywordTok{stan_glm}\NormalTok{(kid_score }\OperatorTok{~}\StringTok{ }\NormalTok{mom_hs }\OperatorTok{+}\StringTok{ }\NormalTok{mom_iq }\OperatorTok{+}\StringTok{ }\NormalTok{noise, }\DataTypeTok{prior=}\KeywordTok{hs}\NormalTok{(),}
                      \DataTypeTok{data=}\NormalTok{kidiqr, }\DataTypeTok{seed=}\NormalTok{SEED, }\DataTypeTok{refresh =} \DecValTok{0}\NormalTok{)}
\KeywordTok{round}\NormalTok{(}\KeywordTok{median}\NormalTok{(}\KeywordTok{bayes_R2}\NormalTok{(fit_3rhs)), }\DecValTok{2}\NormalTok{)}
\end{Highlighting}
\end{Shaded}

\begin{verbatim}
[1] 0.2
\end{verbatim}

\begin{Shaded}
\begin{Highlighting}[]
\KeywordTok{round}\NormalTok{(}\KeywordTok{median}\NormalTok{(}\KeywordTok{loo_R2}\NormalTok{(fit_3rhs)), }\DecValTok{2}\NormalTok{)}
\end{Highlighting}
\end{Shaded}

\begin{verbatim}
[1] 0.2
\end{verbatim}

\begin{Shaded}
\begin{Highlighting}[]
\KeywordTok{round}\NormalTok{(}\KeywordTok{median}\NormalTok{(}\KeywordTok{bayes_R2}\NormalTok{(fit_3rhsn)), }\DecValTok{2}\NormalTok{)}
\end{Highlighting}
\end{Shaded}

\begin{verbatim}
[1] 0.2
\end{verbatim}

\begin{Shaded}
\begin{Highlighting}[]
\KeywordTok{round}\NormalTok{(}\KeywordTok{median}\NormalTok{(}\KeywordTok{loo_R2}\NormalTok{(fit_3rhsn)), }\DecValTok{2}\NormalTok{)}
\end{Highlighting}
\end{Shaded}

\begin{verbatim}
[1] 0.2
\end{verbatim}

\hypertarget{compute-sum-of-log-posterior-predictive-densities}{%
\paragraph{Compute sum of log posterior predictive
densities}\label{compute-sum-of-log-posterior-predictive-densities}}

\texttt{log\_lik} returns \(\log(p(y_i|\theta^{(s)}))\)

\begin{Shaded}
\begin{Highlighting}[]
\NormalTok{ll_}\DecValTok{3}\NormalTok{ <-}\StringTok{ }\KeywordTok{log_lik}\NormalTok{(fit_}\DecValTok{3}\NormalTok{)}
\NormalTok{ll_3n <-}\StringTok{ }\KeywordTok{log_lik}\NormalTok{(fit_3n)}
\end{Highlighting}
\end{Shaded}

compute \(\log(\frac{1}{S}\sum_{s=1}^S p(y_i|\theta^{(s)})\) in
computationally stable way

\begin{Shaded}
\begin{Highlighting}[]
\KeywordTok{round}\NormalTok{(}\KeywordTok{sum}\NormalTok{(matrixStats}\OperatorTok{::}\KeywordTok{colLogSumExps}\NormalTok{(ll_}\DecValTok{3}\NormalTok{) }\OperatorTok{-}\StringTok{ }\KeywordTok{log}\NormalTok{(}\KeywordTok{nrow}\NormalTok{(ll_}\DecValTok{3}\NormalTok{))), }\DecValTok{1}\NormalTok{)}
\end{Highlighting}
\end{Shaded}

\begin{verbatim}
[1] -1872
\end{verbatim}

\begin{Shaded}
\begin{Highlighting}[]
\KeywordTok{round}\NormalTok{(}\KeywordTok{sum}\NormalTok{(matrixStats}\OperatorTok{::}\KeywordTok{colLogSumExps}\NormalTok{(ll_3n) }\OperatorTok{-}\StringTok{ }\KeywordTok{log}\NormalTok{(}\KeywordTok{nrow}\NormalTok{(ll_3n))), }\DecValTok{1}\NormalTok{)}
\end{Highlighting}
\end{Shaded}

\begin{verbatim}
[1] -1871.1
\end{verbatim}

\hypertarget{compute-log-loo-predictive-densities-1}{%
\paragraph{Compute log LOO predictive
densities}\label{compute-log-loo-predictive-densities-1}}

\texttt{loo} uses fast approximate leave-one-out cross-validation

\begin{Shaded}
\begin{Highlighting}[]
\NormalTok{loo_}\DecValTok{3}\NormalTok{ <-}\StringTok{ }\KeywordTok{loo}\NormalTok{(fit_}\DecValTok{3}\NormalTok{)}
\NormalTok{loo_3n <-}\StringTok{ }\KeywordTok{loo}\NormalTok{(fit_3n)}
\KeywordTok{round}\NormalTok{(loo_}\DecValTok{3}\OperatorTok{$}\NormalTok{estimate[}\StringTok{"elpd_loo"}\NormalTok{,}\DecValTok{1}\NormalTok{], }\DecValTok{1}\NormalTok{)}
\end{Highlighting}
\end{Shaded}

\begin{verbatim}
[1] -1876.1
\end{verbatim}

\begin{Shaded}
\begin{Highlighting}[]
\KeywordTok{round}\NormalTok{(loo_3n}\OperatorTok{$}\NormalTok{estimate[}\StringTok{"elpd_loo"}\NormalTok{,}\DecValTok{1}\NormalTok{], }\DecValTok{1}\NormalTok{)}
\end{Highlighting}
\end{Shaded}

\begin{verbatim}
[1] -1879.8
\end{verbatim}

\hypertarget{more-information-on-loo-elpd-estimate-including-standard-errors}{%
\paragraph{More information on LOO elpd estimate including standard
errors}\label{more-information-on-loo-elpd-estimate-including-standard-errors}}

\begin{Shaded}
\begin{Highlighting}[]
\NormalTok{loo_}\DecValTok{3}
\end{Highlighting}
\end{Shaded}

\begin{verbatim}

Computed from 4000 by 434 log-likelihood matrix

         Estimate   SE
elpd_loo  -1876.1 14.2
p_loo         4.0  0.4
looic      3752.1 28.5
------
Monte Carlo SE of elpd_loo is 0.0.

All Pareto k estimates are good (k < 0.5).
See help('pareto-k-diagnostic') for details.
\end{verbatim}

\begin{Shaded}
\begin{Highlighting}[]
\NormalTok{loo_3n}
\end{Highlighting}
\end{Shaded}

\begin{verbatim}

Computed from 4000 by 434 log-likelihood matrix

         Estimate   SE
elpd_loo  -1879.8 14.2
p_loo         8.7  0.7
looic      3759.6 28.4
------
Monte Carlo SE of elpd_loo is 0.0.

All Pareto k estimates are good (k < 0.5).
See help('pareto-k-diagnostic') for details.
\end{verbatim}

\hypertarget{model-using-only-the-maternal-high-school-indicator}{%
\paragraph{Model using only the maternal high school
indicator}\label{model-using-only-the-maternal-high-school-indicator}}

\begin{Shaded}
\begin{Highlighting}[]
\NormalTok{fit_}\DecValTok{1}\NormalTok{ <-}\StringTok{ }\KeywordTok{stan_glm}\NormalTok{(kid_score }\OperatorTok{~}\StringTok{ }\NormalTok{mom_hs, }\DataTypeTok{data=}\NormalTok{kidiq, }\DataTypeTok{refresh =} \DecValTok{0}\NormalTok{)}
\NormalTok{(loo_}\DecValTok{1}\NormalTok{ <-}\StringTok{ }\KeywordTok{loo}\NormalTok{(fit_}\DecValTok{1}\NormalTok{))}
\end{Highlighting}
\end{Shaded}

\begin{verbatim}

Computed from 4000 by 434 log-likelihood matrix

         Estimate   SE
elpd_loo  -1914.8 13.8
p_loo         3.1  0.3
looic      3829.7 27.7
------
Monte Carlo SE of elpd_loo is 0.0.

All Pareto k estimates are good (k < 0.5).
See help('pareto-k-diagnostic') for details.
\end{verbatim}

\hypertarget{compare-models-using-loo-log-score-elpd}{%
\paragraph{Compare models using LOO log score
(elpd)}\label{compare-models-using-loo-log-score-elpd}}

\begin{Shaded}
\begin{Highlighting}[]
\KeywordTok{loo_compare}\NormalTok{(loo_}\DecValTok{3}\NormalTok{, loo_}\DecValTok{1}\NormalTok{)}
\end{Highlighting}
\end{Shaded}

\begin{verbatim}
      elpd_diff se_diff
fit_3   0.0       0.0  
fit_1 -38.8       8.3  
\end{verbatim}

\hypertarget{compare-models-using-loo-r2}{%
\paragraph{Compare models using
LOO-R\^{}2}\label{compare-models-using-loo-r2}}

\begin{Shaded}
\begin{Highlighting}[]
\CommentTok{# we need to fix the seed to get comparison to work correctly in this case}
\KeywordTok{set.seed}\NormalTok{(}\DecValTok{1414}\NormalTok{)}
\NormalTok{looR2_}\DecValTok{1}\NormalTok{ <-}\StringTok{ }\KeywordTok{loo_R2}\NormalTok{(fit_}\DecValTok{1}\NormalTok{)}
\KeywordTok{set.seed}\NormalTok{(}\DecValTok{1414}\NormalTok{)}
\NormalTok{looR2_}\DecValTok{3}\NormalTok{ <-}\StringTok{ }\KeywordTok{loo_R2}\NormalTok{(fit_}\DecValTok{3}\NormalTok{)}
\KeywordTok{round}\NormalTok{(}\KeywordTok{mean}\NormalTok{(looR2_}\DecValTok{1}\NormalTok{), }\DecValTok{2}\NormalTok{)}
\end{Highlighting}
\end{Shaded}

\begin{verbatim}
[1] 0.05
\end{verbatim}

\begin{Shaded}
\begin{Highlighting}[]
\KeywordTok{round}\NormalTok{(}\KeywordTok{mean}\NormalTok{(looR2_}\DecValTok{3}\NormalTok{), }\DecValTok{2}\NormalTok{)}
\end{Highlighting}
\end{Shaded}

\begin{verbatim}
[1] 0.2
\end{verbatim}

\begin{Shaded}
\begin{Highlighting}[]
\KeywordTok{round}\NormalTok{(}\KeywordTok{mean}\NormalTok{(looR2_}\DecValTok{3}\OperatorTok{-}\NormalTok{looR2_}\DecValTok{1}\NormalTok{), }\DecValTok{2}\NormalTok{)}
\end{Highlighting}
\end{Shaded}

\begin{verbatim}
[1] 0.16
\end{verbatim}

\begin{Shaded}
\begin{Highlighting}[]
\KeywordTok{round}\NormalTok{(}\KeywordTok{sd}\NormalTok{(looR2_}\DecValTok{3}\OperatorTok{-}\NormalTok{looR2_}\DecValTok{1}\NormalTok{), }\DecValTok{2}\NormalTok{)}
\end{Highlighting}
\end{Shaded}

\begin{verbatim}
[1] 0.04
\end{verbatim}

\hypertarget{model-with-an-interaction}{%
\paragraph{Model with an interaction}\label{model-with-an-interaction}}

\begin{Shaded}
\begin{Highlighting}[]
\NormalTok{fit_}\DecValTok{4}\NormalTok{ <-}\StringTok{ }\KeywordTok{stan_glm}\NormalTok{(kid_score }\OperatorTok{~}\StringTok{ }\NormalTok{mom_hs }\OperatorTok{+}\StringTok{ }\NormalTok{mom_iq }\OperatorTok{+}\StringTok{ }\NormalTok{mom_hs}\OperatorTok{:}\NormalTok{mom_iq,}
                  \DataTypeTok{data=}\NormalTok{kidiq, }\DataTypeTok{refresh=}\DecValTok{0}\NormalTok{)}
\end{Highlighting}
\end{Shaded}

\hypertarget{compare-models-using-loo-log-score-elpd-1}{%
\paragraph{Compare models using LOO log score
(elpd)}\label{compare-models-using-loo-log-score-elpd-1}}

\begin{Shaded}
\begin{Highlighting}[]
\NormalTok{loo_}\DecValTok{4}\NormalTok{ <-}\StringTok{ }\KeywordTok{loo}\NormalTok{(fit_}\DecValTok{4}\NormalTok{)}
\KeywordTok{loo_compare}\NormalTok{(loo_}\DecValTok{3}\NormalTok{, loo_}\DecValTok{4}\NormalTok{)}
\end{Highlighting}
\end{Shaded}

\begin{verbatim}
      elpd_diff se_diff
fit_4  0.0       0.0   
fit_3 -3.5       2.8   
\end{verbatim}

\hypertarget{compare-models-using-loo-r2-1}{%
\paragraph{Compare models using
LOO-R\^{}2}\label{compare-models-using-loo-r2-1}}

\begin{Shaded}
\begin{Highlighting}[]
\KeywordTok{set.seed}\NormalTok{(}\DecValTok{1414}\NormalTok{)}
\NormalTok{looR2_}\DecValTok{4}\NormalTok{ <-}\StringTok{ }\KeywordTok{loo_R2}\NormalTok{(fit_}\DecValTok{4}\NormalTok{)}
\KeywordTok{round}\NormalTok{(}\KeywordTok{mean}\NormalTok{(looR2_}\DecValTok{4}\NormalTok{), }\DecValTok{2}\NormalTok{)}
\end{Highlighting}
\end{Shaded}

\begin{verbatim}
[1] 0.22
\end{verbatim}

\begin{Shaded}
\begin{Highlighting}[]
\KeywordTok{round}\NormalTok{(}\KeywordTok{mean}\NormalTok{(looR2_}\DecValTok{4}\OperatorTok{-}\NormalTok{looR2_}\DecValTok{3}\NormalTok{), }\DecValTok{2}\NormalTok{)}
\end{Highlighting}
\end{Shaded}

\begin{verbatim}
[1] 0.01
\end{verbatim}

\begin{Shaded}
\begin{Highlighting}[]
\KeywordTok{round}\NormalTok{(}\KeywordTok{sd}\NormalTok{(looR2_}\DecValTok{4}\OperatorTok{-}\NormalTok{looR2_}\DecValTok{3}\NormalTok{), }\DecValTok{2}\NormalTok{)}
\end{Highlighting}
\end{Shaded}

\begin{verbatim}
[1] 0.05
\end{verbatim}

\end{document}
